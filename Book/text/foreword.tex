iOSDC 2018\cite{iosdc.jp:2018}では「iOSエンジニアに聞いて欲しいトーク(30分)」という枠が追加されました.

\begin{quote}
トークのテーマは必ずしもiOS関連の話である必要はありません。iOSエンジニアが聞いて面白ければ何でもアリです!
\end{quote}

\noindent とのことです\cite{fortee.jp:iosdc-japan-2018/speaker/proposal/cfp}.そこで,「フォントと組版の30分入門」とタイトルをつけて次の説明文とともにプロポーザルを出しました\cite{fortee.jp:iosdc-japan-2018/proposal/8e9e8e22-8ff1-4381-813a-347475c2606f}.要するにニッチな方向に攻めました.

\begin{quote}
フォントや組版について気にしたことはありますか? \\
奥深く興味深い世界ですが、そのぶん難しい用語や規則がたくさん。間違えるとこわーい人にツッコミを入れられてしまうかも! \\
本セッションでは、日頃TextKitと親しくしている皆様、技術同人誌に興味のある皆様を対象に、基礎とちょっとした雑学を学びます。
\end{quote}

\noindent 言語実装でもハードウェアでもネットワークでもサーバサイドでもない話題です.怖くなったので強引にTextKitという言葉を出してごまかしています.また,発表の意義を高めるために技術同人誌という言葉も出しました.Twitterには

\begin{quote}
\noindent フォントと組版の30分入門 by S\_Shimotori \textbar プロポーザル \textbar iOSDC Japan 2018 - \href{https://t.co/DZDIy0aFCd}{https://fortee.jp/} \href{https://t.co/zvSrclhYoT}{https://fortee.jp/iosdc-japan-20 \ldots}せっかくのiOSDCなので、WWDCのTextKitのセッションに対抗してDTP方面の話をしたいです!よろしくお願いします!!
\end{quote}

\noindent と投稿して宣伝しました\cite{twitter.com:S_Shimotori_pub/status/1008257832863428609}.せっかくのiOSDCですから,iOSのコアな話をするのが王道でしょうが,そのようなネタは結局最後まで思い浮かばなかったので応募しませんでした.

こうした経緯により,皆様の応援をもって「フォントと組版の30分入門」は無事採択となりました.「iOSエンジニアに聞いて欲しいトーク(30分)」枠かつニッチ狙いですので,iOSで縦書きを試みるような実用的な話は行わないつもりです.よろしくお願いします.

\begin{flushright}
2018年7月14日 \\
しもとりしぐれ
\end{flushright}

