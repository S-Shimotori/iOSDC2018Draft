本章では,iOSで文字列の縦書きを試みる.iOSでの描画には通常UIKitを用いる.UIKitではラベルの描画のために{\sf UILabel}クラスが用意されている.

\begin{lstlisting}[language=swift,caption=Plainモードの{\sf UILabel},label=lst:ios/vertical/uilabel_and_string]
let label = UILabel()
label.backgroundColor = .white
label.textColor = .gray
label.text = "フォントと組版の30分入門"
label.sizeToFit()
\end{lstlisting}

\begin{lstlisting}[language=swift,caption=Attributedモードの{\sf UILabel},label=lst:ios/vertical/uilabel_and_nsattributedstring]
let label = UILabel()
let attributed = NSMutableAttributedString(string: "フォントと組版の30分入門")
attributed.addAttribute(.foregroundColor, value: UIColor.gray, range: NSRange(location: 0, length: 7))
attributed.addAttribute(.backgroundColor, value: UIColor.white, range: NSRange(location: 0, length: 13))
label.attributedText = attributed
label.sizeToFit()
\end{lstlisting}

プログラム\ref{lst:ios/vertical/uilabel_and_string}とプログラム\ref{lst:ios/vertical/uilabel_and_nsattributedstring}は{\sf UILabel}の使用例である.{\sf UILabel}は,{\sf text}プロパティに{\sf String}({\sf NSString})を渡すとPlainモードに,{\sf attributedText}プロパティに{\sf NSAttributedString}を渡すとAttributedモードになる\cite{developer.apple.com:documentation/uikit/uilabel}.
{\sf NSAttributedString}の属性キー{\sf NSAttributedString.Key}には文字列の方向を定める{\sf verticalGlyphForm}があるが,iOSでは常に横書きとされるため値を設定しても縦書きにならない\footnote{{\sf verticalGlyphForm}そのものはiOS 6.0以降で使える.iOS6って2012年なんだけどなぁ.}\cite{developer.apple.com:documentation/foundation/nsattributedstring/key/1528658-verticalglyphform}.

