和文は,金属活字の時代からタテ組の全角ベタ組が基本とされた。タテ組とは,日本語の文字の中心線をタテ方向に揃え並べて版面を構成すること。またヨコ組とは,文字の中心線はタテ向きのまま,アルファベットのように左から右へ中心線をヨコにずらすように並べて版面を構成すること\cite{handbook_of_typography}。
