和文は,金属活字の時代からタテ組の全角ベタ組が基本とされた。タテ組とは,日本語の文字の中心線をタテ方向に揃え並べて版面を構成すること。またヨコ組とは,文字の中心線はタテ向きのまま,アルファベットのように左から右へ中心線をヨコにずらすように並べて版面を構成すること\cite{handbook_of_typography}。

日本語の表記は、漢字・平仮名・片仮名に加えて、ローマ字・アラビア数字など、多くの文字体系を共存させることができ、句読点・括弧などの記述記号を使うという特徴があります\cite{introduction_to_japanese_typesetting}。

日本語の表記において、一般論として"唯一の正書法"というものは定義されていません\cite{introduction_to_japanese_typesetting}。

活字デザインは正方形の仮想ボディを基準としてデザインされているので、ほとんどの漢字・平仮名・片仮名は縦横両用として使用できます——しかし、平仮名・片仮名の字体は縦書きを基本として発達したものなので、活字デザインにおいても、この字体の課題を解決しているとはいえず、横書きに対応する活字デザインの開発が課題となっている\cite{introduction_to_japanese_typesetting}。

日本語組版は、一般には字間にアキを挟まないベタ組みであり、「指定行長を文字サイズである全角の整数倍に指定」します。しかし、全ての行をベタ送りで組めることは、ほとんどありません\cite{introduction_to_japanese_typesetting}。

漢字・仮名の活字デザインは一般に正方形の仮想ボディで構成されており、ベタ組みで各文字が接触しないように、仮想ボディの内側に外接矩形〈字面〉が収まるように設計されています——漢字の一般的な字面は仮想ボディの九〇パーセント。この仮想ボディの寸法が〈文字サイズ〉にあたり、書字方向の寸法を〈字幅〉といいます\cite{introduction_to_japanese_typesetting}。

組版では、字間・幅の量を、実態としてのボディを持つ金属活字に由来する〈全角 〉〈二分にぶん/半角はんかく〉〈三分さんぶん〉〈四分しぶん〉〈八分はちぶん〉を単位として使用します\cite{introduction_to_japanese_typesetting}。

明朝体の特徴\\
漢字は筆文字の特徴をとらえつつ、金属活字時代に効率的に彫刻するために、文字を構成する要素(「てん」「はらい」「始線」「うろこ」など)のデザインが洗練化、様式化されている。ひらがなは、毛筆そのままの印象を漢字よりも強く残している\cite{mdn_201507}

現在使われている多くの日本語フォントには、脈々と受け継がれてきたルーツがある。明朝体は、欧米の活字製作技術とともに伝来して、金属活字が洗練されていくなかで築地体などの名作書体が生まれた。そうした優れた活字書体は、その後の写真写植やデジタルで生まれた新しい書体にも大きな影響を与えている。

活字時代の名作書体の中には、復刻されてデジタルで使えるようになったものもある\cite{mdn_201611}。

日本語書体の主役「明朝体」。「明朝」という名前の通り、中国の書から生まれ、ヨーロッパの活字技術とともに日本に渡ってきた、実はグローバルな書体だ。そこに日本の「かな文字」が加えられて、日本語書体として独自の発展を遂げてきた\cite{mdn_201611}。

文字に注目して新聞を見ると真っ先に気づくのが、本文の書体が縦に潰れている(平体になっている)ことだろう。新聞向けの明朝体は、本文なら80\%前後、平体をかけて用いられることを前提にデザインされている。平体をかけるのは、可読性をキープしながら、できるだけ情報を多く詰め込めるようにする工夫だと言われているが、どうだろうか?\cite{mdn_201611}

新聞向け明朝体のように縦を潰して平たくするのが平体なら、横幅をシェイプして縦長に見せるのが長体\cite{mdn_201611}。
