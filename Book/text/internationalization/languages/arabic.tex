印刷物に関して言えば,早くは10〜11世紀頃にエジブトで木版印刷が行われていた形跡があるが,イスラーム世界ではヨーロッパから活版印刷が伝わった後も,諸々の理由で近代になるまで普及しなかった.18世紀にドイツで発明されたリトグラフは,手書き写本のような字形やレイアウトを表現できるため,特にイランやパキスタンで好まれた。今日ではコンピュータによる組版が普及しているが,イスラーム世界には漢字圏同様の書道の伝統が息づいており,クルアーンやペルシア語の古典詩作品等は,美しい書体で印刷されることが多い。ユニコード標準では,アラビア文字のそれぞれについて,左右の文字と結合するかしないかといった情報を Unicode 文字データベース(UCD)内の ArabicShaping.txt ファイルによって定義している.また,ユニコードにおけるアラビア文字の扱いについて詳しくは,ユニコード標準の規格書の第 9 章を参照\cite{islamic_area_studies_resources}.

アラビア文字の印刷は、中東・イスラム地域よりもはるかに早く、ヨーロッパにおいて開始された。16世紀に入ると、キリスト教の祈祷書など、アラブ人キリスト教徒向けの出版物からアラビア文字の印刷技術は定着し、後には東洋学の発展とともに、地理書、文法書、古典文学など、さまざまな非宗教文献も刊行されるようになった。イスラム教徒が印刷技術の導入に消極的だった原因については様々な理由が挙げられている。例えば、イスラム教徒の伝統的な学問では、知識は師弟相伝の記憶によって伝えられることが重視されていたことが指摘される。18世紀に顕著となったオスマン帝国のヨーロッパ諸国に対する軍事的な弱体化を背景に、ヨーロッパ式の軍事改革と科学技術の導入が不可避となった結果として、オスマン帝国でも、印刷によって書物を大量生産し、知識を普及する必要性がようやく認識されたことになる。文字の美観の問題も見逃すことはできない。19世紀頃までヨーロッパで使用されていた活字はのっぺりとして不格好であり、美しい筆跡の写本を愛好するイスラム教徒の審美眼に適うものではなかった7。イスラム社会の伝統的な学問分野では、100年前のミュテフェッリカの時代には印刷が禁止されていた宗教関係が出版の対象に含まれることが目を引く。ブーラーク以後、19世紀にはイスラム教徒が経営し、アラビア文字で印刷を行う出版事業が中東全域に広がった\cite{rnavi.ndl.go.jp:asia/entry/bulletin7-3-2.php}。
