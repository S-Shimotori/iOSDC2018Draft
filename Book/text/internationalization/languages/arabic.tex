印刷物に関して言えば,早くは10〜11世紀頃にエジブトで木版印刷が行われていた形跡があるが,イスラーム世界ではヨーロッパから活版印刷が伝わった後も,諸々の理由で近代になるまで普及しなかった.18世紀にドイツで発明されたリトグラフは,手書き写本のような字形やレイアウトを表現できるため,特にイランやパキスタンで好まれた。今日ではコンピュータによる組版が普及しているが,イスラーム世界には漢字圏同様の書道の伝統が息づいており,クルアーンやペルシア語の古典詩作品等は,美しい書体で印刷されることが多い。ユニコード標準では,アラビア文字のそれぞれについて,左右の文字と結合するかしないかといった情報を Unicode 文字データベース(UCD)内の ArabicShaping.txt ファイルによって定義している.また,ユニコードにおけるアラビア文字の扱いについて詳しくは,ユニコード標準の規格書の第 9 章を参照\cite{islamic_area_studies_resources}.

アラビア文字の印刷は、中東・イスラム地域よりもはるかに早く、ヨーロッパにおいて開始された。16世紀に入ると、キリスト教の祈祷書など、アラブ人キリスト教徒向けの出版物からアラビア文字の印刷技術は定着し、後には東洋学の発展とともに、地理書、文法書、古典文学など、さまざまな非宗教文献も刊行されるようになった。イスラム教徒が印刷技術の導入に消極的だった原因については様々な理由が挙げられている。例えば、イスラム教徒の伝統的な学問では、知識は師弟相伝の記憶によって伝えられることが重視されていたことが指摘される。18世紀に顕著となったオスマン帝国のヨーロッパ諸国に対する軍事的な弱体化を背景に、ヨーロッパ式の軍事改革と科学技術の導入が不可避となった結果として、オスマン帝国でも、印刷によって書物を大量生産し、知識を普及する必要性がようやく認識されたことになる。文字の美観の問題も見逃すことはできない。19世紀頃までヨーロッパで使用されていた活字はのっぺりとして不格好であり、美しい筆跡の写本を愛好するイスラム教徒の審美眼に適うものではなかった7。イスラム社会の伝統的な学問分野では、100年前のミュテフェッリカの時代には印刷が禁止されていた宗教関係が出版の対象に含まれることが目を引く。ブーラーク以後、19世紀にはイスラム教徒が経営し、アラビア文字で印刷を行う出版事業が中東全域に広がった\cite{rnavi.ndl.go.jp:asia/entry/bulletin7-3-2.php}。

活版印刷の技術がアラビア書道及びアラビア書道家の立場を大きく変化させました。当初受け入れられなかった活版印刷によるクルアーンの印刷も1727年にリトグラフ印刷術が導入され、クルアーンの復元が出来るようになりました。そしてついに1924年になってカイロでクルアーンの活字印刷が許可されるようになりました。更にデジタル化が進むにつれて、アラビア書道はクルアーンの写本という役割から純粋に芸術の分野に移行してゆくようになりました\cite{www.jaca2006.org/アラビア書道とその歴史/}。

アラブ人は紀元前9世紀に歴史文献に登場して以来、メソポタミアの古代文明圏の周縁でゆっくりと自らの社会を成熟させ、遊牧社会とオアシス都市のネットワークの中で言語文化を育んできました。そうした長期の準備期間を経て、ご存知のように、7世紀にはアラビア語で啓示されたクルアーン(コーラン)の教えを中核として、イスラーム世界とその文化が成立発展し、人類文明に多大の貢献を果たしてきました。現代のアラビア語世界は、まさに、この事実に由来する大きな特徴を持っています。アラブの言語文化の中で生み出されたクルアーンの言語は、アラブ人口の何倍ものイスラーム世界の文化を担う役割を果たし、アラブ世界と非アラブ世界に不可欠の文明言語として成長しました。文明言語とは、文字表記による「書き言葉」を意味します。この結果、アラビア語を母語とするアラブ人の社会の中で日々使用される「話し言葉」は、次第に書き言葉と違ったものとなり、現代アラブ社会では、書き言葉と話し言葉がずいぶん異なってしまいました。預言者ムハンマドとほぼ同世代である聖徳太子の頃の日本語(7世紀)と現代日本語が異なることを、私たちは自明のことと受け止めています。英語なら、その記録が始まる前なので、同時代の比べるべき言語記録もありません。18世紀末から始まったアラブ世界の近代化にともない、アラビア語は様々な言語改革を経て今日の「現代標準アラビア語」(書き言葉)として発展してきました。ですから、皆さんはアラブ世界の共通語としてこの言葉を学びます。もちろん、アラブ世界各地域の話し言葉とは大きく異なりますが、皆さんが学ぶアラビア語は、広大なアラブ世界のどこでも理解され尊敬される言葉です。いわば、広いアラブ世界への「国際パスポート」と言えましょう\cite{el.minoh.osaka-u.ac.jp/flc/ara/culture01.html}。

Arabizi is a form of writing Arabic text which relies on using Latin letters rather than Arabic letters. This form of writing is common with the Arab youth\cite{sentiment_analysis_for_arabizi_text}.

アラビア語チャットでは、”Arabizi”と呼ばれるローマ字によるアラビア語表記がよく用いられます。Arabiziは進化を続けながら急速に広まる一方、地域による方言も様々です\cite{www.basistech.jp/rosette-api-adds-support-for-arabizi-script/}。

cf) \href{https://en.wikipedia.org/wiki/Arabic_chat_alphabet}{Arabic chat alphabet - Wikipedia}

確かにグーテンベルク以前の西欧では,写本は修道院などで細々と生産されるにすぎなかった。他方,当時の世界文明の先端を行くイスラーム世界では,紙と手稿本が8世紀末には市場を生みだし,9世紀にはそのような市場に依存して生きる作家も登場した\cite{history_of_islamic_books}。

手稿本/写本のイスラーム文化は,8世紀末頃から始まり,18世紀まで栄え,部分的には近代的な印刷技術が導入された19世紀まで続いた.言いかえると,イスラーム世界ではグーテンベルク的な印刷の導入は遅れた。従来,その原因として,宗教的な保守主義や文化的な後進性が指摘されることが多かったが,本書でも明らかになるように,それは適切な理解ではない\cite{history_of_islamic_books}。

西アジアのイスラーム世界にオスマン朝があった時代,隣りのヨーロッパでは「大航海時代」ならぬ,「大印刷時代」が進行し,本の流通量が飛躍的に増加しつつあった。この技術はすぐにオスマン朝にも伝えられたが,一部の非ムスリムのコミュニティで用いられたに留まり,社会全体で積極的に導入されることはなかった。その理由は様々に議論されてきたが,主な理由は限られた需要にあったのだろう。イスラーム世界では確固たる本の伝統があった上,当時必要とされた本の量は手書きにより生産可能だったからである。オスマン朝において印刷技術が本格的に導入されるのは,本やその他の印刷物に新しい需要の生まれた19世紀初頭のことである\cite{history_of_islamic_books}。

石版技術はまもなくイスラーム世界にも紹介された。イスラーム世界にとっては,活版印刷以上に石版技術が「新技術」であり,より熱心にとりいれられた。その浸透は次の二つの理由による。一つは,19世紀前半に政府系印刷所が印刷した技術書や軍事教練用書籍には,図版が不可欠だったせいである。図版の印刷には石版は最適だった。また石版では手書きの文字をそのまま印刷し,量産することが可能だった。これは,手書き本になじんだイスラーム世界の人々にとってより親しみやすいものだった\cite{history_of_islamic_books}。

16世紀から活動したローマをはじめとするヨーロッパの印刷所で印刷されたアラビア文字の字体の拙さは輸入品としての印刷本の評価を下げ,結果として印刷への関心を失わせる結果を招いたといわれている\cite{history_of_islamic_books}。

マックは多言語を扱っていたから,アラビア語でもマックが先行した。1990年代までは,アラビア語およびアラビア文字を用いる言語の専門家でも,マック派が多かった\cite{history_of_islamic_books}。
