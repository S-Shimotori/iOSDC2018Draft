\begin{description}
    \item[space(U+0020)]
    \item[no-break space(U+00A0)]
\end{description}

伝統的な組版では単語間のスペースに複数の文字を使用していた.デジタルフォントでは,word spaceとは単語間スペースに用いられる特定の幅の文字である\cite{docs.microsoft.com:en-us/typography/develop/character-design-standards/whitespace}.

en dash\cite{handbook_of_typography}

\begin{itemize}
    \item 括弧のようにダーシを対にして語の前後に入れ,強調する。
    \item 日時の期間,年齢の幅,ページ,値段などの範囲。
    \item 対局のもの,地点と地点,人と人など,関係や連結。
\end{itemize}

em dash\cite{handbook_of_typography}

\begin{itemize}
    \item 文と文の間,字句と字句の間に用いて,時間の経過を表す。
    \item 括弧のようにダーシを対にして文を囲み,説明や副題などを示す。
    \item 行頭に用いて,引用を表す。
    \item 語尾に用いて省略を表す。
\end{itemize}

マイナス記号は,通常キーボードで打ちやすいhyphen-minus[ハイフンマイナス]が使われているが,それとは別に減算記号としてminus sign[マイナス記号]があるので,数学の組版では注意が必要。なお,多くのプログラム等では,演算における減算記号はマイナス記号でなくハイフンマイナスが使用される\cite{handbook_of_typography}。
