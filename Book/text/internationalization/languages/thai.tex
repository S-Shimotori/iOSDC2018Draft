タイ語はタイ王国の公用語.タイ全土で話されている.使用人口はタイ国内に約6200万人\cite{www.coelang.tufs.ac.jp:mt/th/}.

ひとつの音節が,子音文字,母音記号,声調記号などの記号の組み合わせで表示される.子音字を中心に母音記号を上下左右に配置する音節文字である.子音字の文字幅は文字によって大きく異なる(=プロポーショナル).発音順とグリフの並び順は一致せず,音素と文字のグリフが1対1対応でないことがある\cite{www.aa.tufs.ac.jp:mmine/lecture/lec03/TLKChar/lecTLK03.htm}.

通常,タイ文字組版の文章内ではスペースを空けずに組むが,アラビア数字や英数字が文中に入るときは前後にスペースを入れる.組み方向は横組み,行揃えは均等配置(最終行左揃え)が基本だが左揃えやセンター揃えが使われることもある.単語の間では改行を行わない(ハイフネーションに相当するものがない)ので改行自体が少ない方が好まれる.文字の上下に記号がつくので十分に行間をとる必要がある.公文書ではインデントを行って組むこととされる.句読点のような文字はない\cite{www.morisawa.co.jp:fonts/multilingual/typesetting/pdf/all_multilingualtypesetting_1802.pdf}.
