用語(あとで消す)

モリサワとW3C\cite{www.w3.org:TR/2011/WD-jlreq-20111129/ja/}あたりから引用中.

\begin{description}
    \item[Advancement]
    \item[Alignment] alignmentとdirectionalityは異なるものを指すので注意\cite{developer.apple.com:videos/play/wwdc2016/232/}.
    \item[Ascent]
    \item[Baseline(並び線)]
    \item[base writing direction] 最初に登場したstrong characterで決定される\cite{developer.apple.com:videos/play/wwdc2016/232/}\cite{unicode.org:reports/tr9/}.
    \item[bidirectional algorithm(双方向アルゴリズム)] bidirectional textを扱うためのアルゴリズム\cite{www.w3.org:International/articles/inline-bidi-markup/uba-basics}.bidi algorithmとも.このアルゴリズムによる脆弱性が見つかったことがある\cite{jvndb.jvn.jp:ja/contents/2010/JVNDB-2010-002420.html}.
    \item[bidirectional text(双方向テキスト)] 左横書き(LTR)と右横書き(RTL)が混在するテキストのこと\cite{www.w3.org:International/articles/inline-bidi-markup/uba-basics}.bidiとも.
    \item[block direction(行送り方向)]
    \item[bold(太字)]
    \item[Boldface]
    \item[Bounding rectangle]
    \item[break (a line)(分割)]
    \item[Cap height]
    \item[centering(中央そろえ)]
    \item[character advance(字幅)]
    \item[character frame(外枠)]
    \item[characters not starting line(行頭禁則文字)]
    \item[CJK] 中国日本韓国,特に中国語・日本語・韓国語のこと.UnicodeのブロックにもCJKが散見される\cite{unicode.org:Public/UNIDATA/Blocks.txt}.
    \item[composition(組版)]
    \item[descender line(ディセンダライン)]
    \item[Descent]
    \item[DTP(desktop publishing)]1984年1月、Apple社から初代Macintoshが発売される。プラットフォームとして様々な周辺機器やソフトウェアが生み出された。ただし初期のMacは本格的なDTPを行うにはスペックが厳しく、DTP業界が急拡大するのは1987年発売のMacintosh II頃からである。1985年5月、Apple純正のレーザープリンターであるApple LaserWriterが発売される。LaserWriterプリンターは、アドビ社の開発したページ記述言語・PostScript技術を用いた「Adobe PostScriptフォント」がROMメモリに組み込まれており、これによって画面に表示されているものをそのままに印刷することが可能となる「WYSIWYG」を実現したほか、プリンターにPostScriptフォントを搭載している限りはコンピュータとプリンターの組み合わせが変わっても出力結果を維持するという「デバイスインディペンデント」(使用機器に依存しない)な性質を実現していた。1985年7月、Macintoshプラットフォームにおける最初の実用的なDTPアプリケーションとなるアルダス社のPagemakerが発売される。これによってDTP環境が実現された。(ここまでウィキペディア.要出典.新詳説DTP基礎)
    \item[Emoji(絵文字)]
    \item[even inter-character spacing(均等割り)]
    \item[even tsumegumi(均等詰め)]
    \item[face tsumegumi(字面詰め)]
    \item[fixed inter-character spacing(アキ組)]字間に一定のスペースを挟む組版様式.四分アキ,二分アキなどがある.ベタ組に対して言う\cite{www.jpp.co.jp:yougo/a1.html}.
    \item[fixed-width(モノスペース)]
    \item[Font(フォント)]
    \item[Font fallback]
    \item[Font Family]
    \item[Fount]同一ボディとデザインの文字・記号のグループ\cite{www.f.waseda.jp/yukis/hpb/hpb2006.1.html}
\item[Fount scheme(フォントスキーム)]一文字一文字バラバラの活字を組んでいた金属活字時代では,アルファベットの活字はキャラクターひと揃えで販売されていた。各キャラクターの活字の本数の割合は活字鋳造所ごとに決められており,その本数の配当表をフォントスキームという。このフォントスキームにより決定した活字一揃えをfount(UK)といい,現在一般にいわれているfount(US)はこれに由来する\cite{handbook_of_typography}。
    \item[full-width(全角)]
    \item[Glyph(グリフ)]字体とほぼ同義語ですが、記述記号やスペースなども含めたものを指します。慣用的にはデータとしての字体を指す場合に使われることもあります。これらの文字と記号類を集めたものがグリフセットと呼ばれるもので、これは文字セットや文字コレクションとほぼ同義と考えてよいでしょう。 A glyph is the smallest displayable unit in a font. A glyph may represent one character, more than one character, or part of a character. The mapping of characters to glyphs is not simple—it can be many-to-many. In addition, the order and position of glyphs in a line is complex\cite{developer.apple.com:library/archive/documentation/MacOSX/Conceptual/BPInternational/InternationalizingYourCode/InternationalizingYourCode.html}.今日使用している「文字関係のイメージ」すべてをグリフという。文字は時代を経て,文章を組み立てるために人々が多くの候補から様々に使用することによって,変化し,淘汰され,限定されたかたちとなった。グリフの分類として,ラテンアルファベットの他に,句読点,通貨記号,数字および数字記号,分音符記号つきラテン文字,ギリシャ文字,キリル文字などがある\cite{handbook_of_typography}。表記体系上必要な句読点や括弧類、スペースなど、「意味や音を持たない図形記号などの抽象化を含めたもの」と文字体系を合わせて〈グリフ glyph〉といいます。グリフは字体とほぼ同義語ですが対象とする範囲が異なります\cite{introduction_to_japanese_typesetting}。
    \item[half em(二分)]
    \item[half em space(二分アキ)]
    \item[half-width]
    \item[horizontal writing mode(横組)]
    \item[Hyphenation(ハイフネーション)]
    \item[isolates support]\cite{developer.apple.com:videos/play/wwdc2016/232/}
    \item[Italic angle]
    \item[inline direction(字詰め)]
    \item[inseparable characters rule(分離禁止)]
    \item[JIS2004]
    \item[JIS X 0213:2004]
    \item[JIS文字セット]
    \item[Kerning]隣り合った二つの文字間を調節するもので,かつての切り貼りでの文字詰めと同じ機能\cite{handbook_of_typography}。
    \item[Leading]
    \item[letter face(字面)]
    \item[letterpress printing(活字組版)]
    \item[letter spacing]一つ一つの文字と文字の間のこと。一般に間隔を詰めると単語の黒みが増し,空きすぎるとまとまりがなくなり読みにくくなる。通常,本文はあらかじめ決められたnormal setの設定で組むことが多い。ノーマルセットのアキは「0」で「ベタ」ともいい,そこを基準にunitの数値でアキやツメを指定できるともいい,そこを基準にunitの数値でアキやツメを指定できる\cite{handbook_of_typography}。
    \item[Ligature(合字)]隣り合った二つの文字が脈略のあるかたちとして自然にみえるようできたグリフ.ASCIIコードのグリフにある,常識的に使用すべきリガチャーはfi,fl,ffのspellが出たとき。これは,金属活字ではこのような組み合わせまたはこれらを複合させた組み合わせの場合,単独の文字で組むとkernedされた部分と接したり重なってしまうので,連字となった文字を使用していたためである。現在はデジタルなので文字が重なったところが異様なかたちとして残るだけであるが,リガチャーに差し替えることが望ましい\cite{handbook_of_typography}。
    \item[line adjustment(行の調整処理)]
    \item[line adjustment by hanging punctuation(ぶら下げ組)]
    \item[line adjustment by inter-character space expansion(追出し処理)] 組版では,行頭・行末・分離禁止にかかった文字・約物を調整するために次行の字間・約物の前後を詰めて移動させることを指す.字間を調整することになる\cite{www.jpp.co.jp:yougo/a1.html}.
    \item[line adjustment by inter-character space reduction(追込み処理)]
    \item[Line breaking]
    \item[line breaking rules(禁則処理)]
    \item[line end(行末)]
    \item[line end alignment(行末そろえ)]
    \item[line head indent(字下げ)]
    \item[line-end prohibition rule(行末禁則文字)]
    \item[line feed(行送り)]
    \item[Line gap(行間)]
    \item[line head(行頭)]
    \item[line head alignment(行頭そろえ)]
    \item[Line height]
    \item[line length(行長)]
    \item[line-start prohibition rule(行頭禁則)]
    \item[lorem ipsum]略してlipsumともいう。38言語に対応しているダミーテキストで,文章の内容に集中しないようにあえて意味をもたない文章になっている。もともとギリシャ語が引き合いに出されていたことから,このような意味をもたない文章をgreekingというが,lipsumはその典型的なもの。Aldus PageMakerのテンプレートに使用されて以降,ほとんどこれが使用されている\cite{handbook_of_typography}。Lorem Ipsum is simply dummy text of the printing and typesetting industry. Lorem Ipsum has been the industry's standard dummy text ever since the 1500s, when an unknown printer took a galley of type and scrambled it to make a type specimen book. It has survived not only five centuries, but also the leap into electronic typesetting, remaining essentially unchanged. It was popularised in the 1960s with the release of Letraset sheets containing Lorem Ipsum passages, and more recently with desktop publishing software like Aldus PageMaker including versions of Lorem Ipsum\cite{www.lipsum.com}.
    \item[matrix(母型)]
    \item[mixed text composition(混植)]
    \item[Monospace(等幅)]
    \item[new column(改段)]
    \item[new recto(改丁)]
    \item[number of characters per line(字詰め)]
    \item[one em space(全角アキ)]
    \item[one third em(三分)]
    \item[one third em space(三分アキ)]
    \item[Open Type Font]
    \item[Orphan avoidance]
    \item[page break(改ページ)]
    \item[paragraph(段落)]
    \item[paragraph break(改行)]
    \item[paragraph format(段落整形)]
    \item[PDF]
    \item[point]通常はPointという単位を使用する。Pointは歴史的に数々あり,一般に日本で多く使用されているJohnson Pica Point,アメリカで古く使用されたHawks Point,ヨーロッパで使用されるDidot Pointなどがある。また,フランスのPierre Simon Fournierが基礎を作ったポイントシステムの一部が残っている。現在,通常DTPに使用されているPostScript Pointは,1985年にAdobe社とApple社により決議されたもので,国際的に基準となっている\cite{handbook_of_typography}。
    \item[Post Script]
    \item[printing types(活字)]
    \item[proportional(プロポーショナル)]
    \item[quarter em(四分)]
    \item[quarter em space(四分アキ)]
    \item[quarter em width(四分角)]
    \item[Serif]
    \item[side bearing]
    \item[single line alignment method(そろえ)]
    \item[solid setting(ベタ組)]
    \item[space(アキ)]活版印刷の時代,ワードスペースやラインスペースなど印刷されない隙間を作るために,スペースやインテルなどの込めものが使われていた。込めものは活字の高さよりも低く作られているため印刷されない\cite{handbook_of_typography}。
    \item[stem]
    \item[tab setting(タブ処理)]
    \item[tall script]アクセント記号や声調記号の関係でline heightが他の言語の文字より高くなっているもの.文字全体が表示されるように余裕を持っておく必要がある\cite{developer.apple.com:videos/play/wwdc2016/201/}.この手の文字を太字にすると重くなってしまうので太字にはしないほうがいいらしい\cite{material.io:design/typography/language-support.html}.
    \item[text direction(組方向)]
    \item[Tightening]
    \item[Tracking]文字列全体あるいは特定の行だけを一定感覚に文字間を詰める[または空ける]場合に使用する,等間隔の字送り処理機能\cite{handbook_of_typography}。
    \item[True Type Font]
    \item[Truncation]
    \item[Typeface(書体)]
    \item[type-picking(文選)]
    \item[typesetting(組版)] 原稿に基づいて活字を組み,最終的には版を作るまでの作業および完成した版.手動,機械さらにはコンピュータであっても,組版という表現は使われている.手動,機械さらにはコンピュータであっても,組版という表現は使われている\cite{lis_dictionary}.1970年代からはコンピュータによる組版が普及した\cite{www.printing-museum.org:communication/column/pdf/column_6.pdf}.
    \item[Typography]
    \item[unbreakable characters rule(分割禁止)]
    \item[unit]すべてのレターフォームはそれぞれの字幅をもつ。金属活字の時代はすべて違っていたといっても過言ではないが,機械化されるにつれ,字幅はunitの倍数に限定され分類されてきた。写植時代に,emを全角として18unitに分割し,すべてのキャラクターを縦割りにして4から18unitに当てはめた。文字の大きさに関わらず,このunitで字幅に対して字間,語間を決定していく\cite{handbook_of_typography}。
    \item[vertical writing(縦書き)] 日本語ではおなじみ縦書き.アジア圏でよく見られるっぽい.top to bottomとも\cite{eikaiwa.dmm.com:uknow/questions/29852/}\cite{www.w3.org:International/questions/qa-scripts}.CSSの{\sf writing-mode}プロパティでは,CJK用に{\sf vertical-rl}が,モンゴル語用に{\sf vertical-lr}が用意されている\cite{www.w3.org:International/articles/vertical-text/}.
    \item[vertical writing mode(縦組)]
    \item[Weight(ウェイト)]
    \item[widow(ウィドウ)]
    \item[widow adjustment(段落末尾処理)]
    \item[WYSIWYG]
    \item[X-height]
    \item[かべ]
    \item[基本版面]
    \item[ゴシック体]
    \item[字形]JIS X 0208では〈字形〉を、「手書き、印字、画面表示などによって実際に図形として表現したもの」と定義しており、字体を目に見える形にしたものを指します\cite{introduction_to_japanese_typesetting}。
    \item[字種]
    \item[字体]JIS X 0208では〈字体〉を「図形文字の図形表現としての形状についての抽象的概念」と定義しています。これは視覚化する前の文字の骨組みの抽象概念のことを指します\cite{introduction_to_japanese_typesetting}。
    \item[字取り]
    \item[字面]
    \item[字面枠]
    \item[書体]字形に一定の様式・デザインを施したものを〈書体〉といいます\cite{introduction_to_japanese_typesetting}。ある一貫した様式でデザインされた文字の集まりを「書体」という\cite{mdn_201507}。
    \item[縦中横]
    \item[詰め組]
    \item[天付き]
    \item[同行見出し]
    \item[表記体系]文字体系に加えて、正書法、句読法、字体・語句などの選択規則や基準などの言語的慣習を含む文字使用の体系を〈表記体系 writing system〉と呼びます——文字体系・書記系・書字系・書字システムとも。同じ文字体系を用いていても、異なる言語では表記体系に違いが見られることがあります\cite{introduction_to_japanese_typesetting}。
    \item[フォント]同じ書体でも作り手、時代によって個性が違い、この時代を「書風」ということがありますが、これが一般に〈フォント font〉と呼ばれているものです。フォントは、本来「同じサイズで、デザインの同じ活字のひと揃い(大文字・小文字・イタリック・スモールキャップ・数字など)」を指しますが、現在は書体データの意味で使用されています——このフォントが指すそれぞれを書体ということもある\cite{introduction_to_japanese_typesetting}。もともとフォントとは、同じ書体、サイズの欧文の金属活字のワンセットのことを指す。が、今は欧文、日本語書体ともにフォントはデータ化されている\cite{mdn_201507}。
    \item[プロポーショナルメトリクス]
    \item[ペアカーニング情報]
    \item[ベタ組み]
    \item[丸ゴシック体]
    \item[明朝体]
    \item[文字体系]文字において発生的、歴史的な一つのまとまりを形作る集合を〈文字体系 script〉と呼びます——書記系・用字系とも。文字体系はアラビア文字、キリル文字、ラテン文字、漢字の表記や使用法のように、言語と一対一に対応しないことがあり、複数の言語で使用される文字体系が多数存在します\cite{introduction_to_japanese_typesetting}。
\end{description}

