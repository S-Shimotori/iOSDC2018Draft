用語(あとで消す)

モリサワとW3C\cite{www.w3.org:TR/2011/WD-jlreq-20111129/ja/}あたりから引用中.

\begin{description}
    \item[Advancement]
    \item[Alignment] alignmentとdirectionalityは異なるものを指すので注意\cite{developer.apple.com:videos/play/wwdc2016/232/}.
    \item[Ascent]
    \item[Baseline(並び線)]
    \item[base writing direction] 最初に登場したstrong characterで決定される\cite{developer.apple.com:videos/play/wwdc2016/232/}\cite{unicode.org:reports/tr9/}.
    \item[bidirectional algorithm(双方向アルゴリズム)] bidirectional textを扱うためのアルゴリズム\cite{www.w3.org:International/articles/inline-bidi-markup/uba-basics}.bidi algorithmとも.このアルゴリズムによる脆弱性が見つかったことがある\cite{jvndb.jvn.jp:ja/contents/2010/JVNDB-2010-002420.html}.
    \item[bidirectional text(双方向テキスト)] 左横書き(LTR)と右横書き(RTL)が混在するテキストのこと\cite{www.w3.org:International/articles/inline-bidi-markup/uba-basics}.bidiとも.
    \item[block direction(行送り方向)]
    \item[bold(太字)]
    \item[Boldface]
    \item[Bounding rectangle]
    \item[break (a line)(分割)]
    \item[Cap height]
    \item[centering(中央そろえ)]
    \item[character advance(字幅)]
    \item[character frame(外枠)]
    \item[characters not starting line(行頭禁則文字)]
    \item[CJK] 中国日本韓国,特に中国語・日本語・韓国語のこと.UnicodeのブロックにもCJKが散見される\cite{unicode.org:Public/UNIDATA/Blocks.txt}.
    \item[composition(組版)]
    \item[descender line(ディセンダライン)]
    \item[Descent]
    \item[DTP(desktop publishing)]1984年1月、Apple社から初代Macintoshが発売される。プラットフォームとして様々な周辺機器やソフトウェアが生み出された。ただし初期のMacは本格的なDTPを行うにはスペックが厳しく、DTP業界が急拡大するのは1987年発売のMacintosh II頃からである。1985年5月、Apple純正のレーザープリンターであるApple LaserWriterが発売される。LaserWriterプリンターは、アドビ社の開発したページ記述言語・PostScript技術を用いた「Adobe PostScriptフォント」がROMメモリに組み込まれており、これによって画面に表示されているものをそのままに印刷することが可能となる「WYSIWYG」を実現したほか、プリンターにPostScriptフォントを搭載している限りはコンピュータとプリンターの組み合わせが変わっても出力結果を維持するという「デバイスインディペンデント」(使用機器に依存しない)な性質を実現していた。1985年7月、Macintoshプラットフォームにおける最初の実用的なDTPアプリケーションとなるアルダス社のPagemakerが発売される。これによってDTP環境が実現された。(ここまでウィキペディア.要出典.新詳説DTP基礎)
    \item[Emoji(絵文字)]
    \item[even inter-character spacing(均等割り)]
    \item[even tsumegumi(均等詰め)]
    \item[face tsumegumi(字面詰め)]
    \item[fixed inter-character spacing(アキ組)]
    \item[fixed-width(モノスペース)]
    \item[Font(フォント)]
    \item[Font fallback]
    \item[Font Family]
    \item[full-width(全角)]
    \item[Glyph(グリフ)]字体とほぼ同義語ですが、記述記号やスペースなども含めたものを指します。慣用的にはデータとしての字体を指す場合に使われることもあります。これらの文字と記号類を集めたものがグリフセットと呼ばれるもので、これは文字セットや文字コレクションとほぼ同義と考えてよいでしょう。 A glyph is the smallest displayable unit in a font. A glyph may represent one character, more than one character, or part of a character. The mapping of characters to glyphs is not simple—it can be many-to-many. In addition, the order and position of glyphs in a line is complex\cite{developer.apple.com:library/archive/documentation/MacOSX/Conceptual/BPInternational/InternationalizingYourCode/InternationalizingYourCode.html}.
    \item[half em(二分)]
    \item[half em space(二分アキ)]
    \item[half-width]
    \item[horizontal writing mode(横組)]
    \item[Hyphenation(ハイフネーション)]
    \item[isolates support]\cite{developer.apple.com:videos/play/wwdc2016/232/}
    \item[Italic angle]
    \item[inline direction(字詰め)]
    \item[inseparable characters rule(分離禁止)]
    \item[JIS2004]
    \item[JIS X 0213:2004]
    \item[JIS文字セット]
    \item[Kerning]
    \item[Leading]
    \item[letter face(字面)]
    \item[letterpress printing(活字組版)]
    \item[Ligature]
    \item[line adjustment(行の調整処理)]
    \item[line adjustment by hanging punctuation(ぶら下げ組)]
    \item[line adjustment by inter-character space expansion(追出し処理)]
    \item[line adjustment by inter-character space reduction(追込み処理)]
    \item[Line breaking]
    \item[line breaking rules(禁則処理)]
    \item[line end(行末)]
    \item[line end alignment(行末そろえ)]
    \item[line head indent(字下げ)]
    \item[line-end prohibition rule(行末禁則文字)]
    \item[line feed(行送り)]
    \item[Line gap(行間)]
    \item[line head(行頭)]
    \item[line head alignment(行頭そろえ)]
    \item[Line height]
    \item[line length(行長)]
    \item[line-start prohibition rule(行頭禁則)]
    \item[matrix(母型)]
    \item[mixed text composition(混植)]
    \item[Monospace(等幅)]
    \item[new column(改段)]
    \item[new recto(改丁)]
    \item[number of characters per line(字詰め)]
    \item[one em space(全角アキ)]
    \item[one third em(三分)]
    \item[one third em space(三分アキ)]
    \item[Open Type Font]
    \item[Orphan avoidance]
    \item[page break(改ページ)]
    \item[paragraph(段落)]
    \item[paragraph break(改行)]
    \item[paragraph format(段落整形)]
    \item[PDF]
    \item[Post Script]
    \item[printing types(活字)]
    \item[proportional(プロポーショナル)]
    \item[quarter em(四分)]
    \item[quarter em space(四分アキ)]
    \item[quarter em width(四分角)]
    \item[Serif]
    \item[single line alignment method(そろえ)]
    \item[solid setting(ベタ組)]
    \item[space(アキ)]
    \item[stem]
    \item[tab setting(タブ処理)]
    \item[tall script]アクセント記号や声調記号の関係でline heightが他の言語の文字より高くなっているもの.文字全体が表示されるように余裕を持っておく必要がある\cite{developer.apple.com:videos/play/wwdc2016/201/}.この手の文字を太字にすると重くなってしまうので太字にはしないほうがいいらしい\cite{material.io:design/typography/language-support.html}.
    \item[text direction(組方向)]
    \item[Tightening]
    \item[Tracking]
    \item[True Type Font]
    \item[Truncation]
    \item[Typeface(書体)]
    \item[type-picking(文選)]
    \item[typesetting(組版)] 原稿に基づいて活字を組み,最終的には版を作るまでの作業および完成した版.手動,機械さらにはコンピュータであっても,組版という表現は使われている.手動,機械さらにはコンピュータであっても,組版という表現は使われている\cite{lis_dictionary}.1970年代からはコンピュータによる組版が普及した\cite{www.printing-museum.org:communication/column/pdf/column_6.pdf}.
    \item[Typography]
    \item[unbreakable characters rule(分割禁止)]
    \item[vertical writing(縦書き)] 日本語ではおなじみ縦書き.アジア圏でよく見られるっぽい.top to bottomとも\cite{eikaiwa.dmm.com:uknow/questions/29852/}\cite{www.w3.org:International/questions/qa-scripts}.CSSの{\sf writing-mode}プロパティでは,CJK用に{\sf vertical-rl}が,モンゴル語用に{\sf vertical-lr}が用意されている\cite{www.w3.org:International/articles/vertical-text/}.
    \item[vertical writing mode(縦組)]
    \item[Weight(ウェイト)]
    \item[widow(ウィドウ)]
    \item[widow adjustment(段落末尾処理)]
    \item[WYSIWYG]
    \item[X-height]
    \item[かべ]
    \item[基本版面]
    \item[ゴシック体]
    \item[字形]
    \item[字体]
    \item[字取り]
    \item[字面]
    \item[字面枠]
    \item[書体]
    \item[縦中横]
    \item[詰め組]
    \item[天付き]
    \item[同行見出し]
    \item[フォント]
    \item[プロポーショナルメトリクス]
    \item[ペアカーニング情報]
    \item[ベタ組み]
    \item[丸ゴシック体]
    \item[明朝体]
\end{description}

