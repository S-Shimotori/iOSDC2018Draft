\documentclass[10pt,twoside,openright,dvipdfmx]{jsbook}

%------------------------------%
% geometry
%------------------------------%

%\usepackage[pass,showframe]{geometry}
\usepackage[pass]{geometry}

%------------------------------%
% afterpage
%------------------------------%

\usepackage{afterpage}
\newcommand\blankpage{%
    \null
    \thispagestyle{empty}
    \addtocounter{page}{-1}
    \newpage
}

%------------------------------%
% mathtools
%------------------------------%

\usepackage{mathtools}

%------------------------------%
% makeidx
%------------------------------%

\usepackage{imakeidx}
\makeindex
\indexsetup{othercode={\thispagestyle{plain}}}

%------------------------------%
% fancyhdr
%------------------------------%

\usepackage{fancyhdr}

\pagestyle{fancy}
\fancyhf{}
\fancyhead[RO]{\leftmark}
\fancyhead[LE]{\rightmark}
\fancyfoot[LE,RO]{\thepage}
\setlength{\footskip}{12pt}

% TODO: apply beginning pages of chapter
\fancypagestyle{plain}{%
    \fancyhf{}
    \fancyfoot[LE,RO]{\thepage}
}

%------------------------------%
% color
%------------------------------%

\usepackage{color}
%http://www.biwako.shiga-u.ac.jp/sensei/kumazawa/tex/color.html
\input{colors/jpncolor}

%------------------------------%
% graphicx
%------------------------------%

\usepackage{graphicx}

%------------------------------%
% listings
%------------------------------%

\usepackage{listings,jlisting}
\lstdefinelanguage{swift}{
    sensitive=true,
    alsoletter={\#},
    keywords=[1]{associatedtype,class,deinit,enum,extension,fileprivate,func,import,init,inout,internal,let,open,operator,private,protocol,public,static,struct,subscript,typealias,var,break,case,continue,default,defer,do,else,fallthrough,for,guard,if,in,repeat,return,switch,where,while,as,Any,catch,false,is,nil,rethrows,super,self,Self,throw,throws,true,try,\_,associativity,convenience,dynamic,didSet,final,get,infix,indirect,lazy,left,mutating,none,nonmutating,optional,override,postfix,precedence,prefix,Protocol,required,right,set,Type,unowned,weak,willSet},
    keywords=[2]{\#available,\#colorLiteral,\#column,\#else,\#elseif,\#endif,\#file,\#fileLiteral,\#function,\#if,\#imageLiteral,\#line,\#selector,\#sourceLocation,@autoclosure},
    numbers=left,
    numberstyle=\scriptsize,
    stepnumber=1,
    numbersep=8pt,
    showstringspaces=false,
    breaklines=true,
    morecomment=[l]{//},
    morecomment=[s]{/*}{*/},
    keywordstyle=[1]\color{konpeki},
    keywordstyle=[2]\color{rindo},
    commentstyle=\color{usumoegi},
    stringstyle=\color{enji},
    morestring=[b]",%"
}

\renewcommand{\lstlistingname}{プログラム}
\lstset{language=swift,
    basicstyle=\ttfamily\scriptsize,
    commentstyle=\textit,
    classoffset=1,
    keywordstyle=\bfseries,
    frame=tRBl,
    framesep=5pt,
    showstringspaces=false,
    numbers=left,
    stepnumber=1,
    numberstyle=\tiny,
    tabsize=2
}



%------------------------------%
% box
%------------------------------%

\usepackage{ascmac,fancybox}

%------------------------------%
% appendix
%------------------------------%

\usepackage[toc,page]{appendix}

\makeatletter
\AtBeginEnvironment{appendices}{%
  \clearpage%
  \fancypagestyle{plain}{%
    \fancyhf{}%
    \renewcommand{\headrulewidth}{0pt}
  }
  \thispagestyle{plain}
}

%------------------------------%
% biblatex
%------------------------------%

\usepackage[style=numeric,backend=biber,defernumbers=true]{biblatex}
\addbibresource{bibliographies/iosdc.bib}
\addbibresource{bibliographies/apple.bib}
\addbibresource{bibliographies/book.bib}
\addbibresource{bibliographies/other.bib}

%------------------------------%
% accented letters
%------------------------------%

\usepackage[utf8]{inputenc}

%------------------------------%
% development
%------------------------------%

\usepackage{lipsum}
\usepackage{layout}

%------------------------------%
% title
%------------------------------%

\usepackage{authblk}

\title{%
    iOSDC 2018 Draft \\
    \large for \href{https://fortee.jp/iosdc-japan-2018/proposal/8e9e8e22-8ff1-4381-813a-347475c2606f}{フォントと組版の30分入門}}
\author[$\dagger$]{しもとりしぐれ}
\affil[$\dagger$]{@S\_Shimotori}
\date{\today}

%------------------------------%
% hyperref
%------------------------------%

\usepackage[colorlinks]{hyperref}

\hypersetup{
    urlcolor=hujinezumi,
    citecolor=sora,
    linkcolor=black
}

\usepackage[normalem]{ulem}

\makeatletter
\begingroup
    \catcode`\$=6
    \catcode`\#=12
    \gdef\href@split$1#$2#$3\\$4{
        \hyper@@link{$1}{$2}{\uline{$4}}
        \endgroup
    }%
\endgroup

%------------------------------%
% document
%------------------------------%

\begin{document}

\pagenumbering{Alph}
\newgeometry{hmarginratio=1:1}
\maketitle
\restoregeometry

\afterpage{\blankpage}

\pagenumbering{roman}

\tableofcontents
\thispagestyle{plain}

\chapter*{はしがき}
\thispagestyle{plain}
\addcontentsline{toc}{chapter}{はしがき}
iOSDC 2018\cite{iosdc.jp:2018}では「iOSエンジニアに聞いて欲しいトーク(30分)」という枠が追加されました.

\begin{quote}
トークのテーマは必ずしもiOS関連の話である必要はありません。iOSエンジニアが聞いて面白ければ何でもアリです!
\end{quote}

\noindent とのことです\cite{fortee.jp:iosdc-japan-2018/speaker/proposal/cfp}.そこで,「フォントと組版の30分入門」とタイトルをつけて次の説明文とともにプロポーザルを出しました\cite{fortee.jp:iosdc-japan-2018/proposal/8e9e8e22-8ff1-4381-813a-347475c2606f}.要するにニッチな方向に攻めました.

\begin{quote}
フォントや組版について気にしたことはありますか? \\
奥深く興味深い世界ですが、そのぶん難しい用語や規則がたくさん。間違えるとこわーい人にツッコミを入れられてしまうかも! \\
本セッションでは、日頃TextKitと親しくしている皆様、技術同人誌に興味のある皆様を対象に、基礎とちょっとした雑学を学びます。
\end{quote}

\noindent 言語実装でもハードウェアでもネットワークでもサーバサイドでもない話題です.怖くなったので強引にTextKitという言葉を出してごまかしています.また,発表の意義を高めるために技術同人誌という言葉も出しました.Twitterには

\begin{quote}
\noindent フォントと組版の30分入門 by S\_Shimotori \textbar プロポーザル \textbar iOSDC Japan 2018 - \href{https://t.co/DZDIy0aFCd}{https://fortee.jp/} \href{https://t.co/zvSrclhYoT}{https://fortee.jp/iosdc-japan-20 \ldots}せっかくのiOSDCなので、WWDCのTextKitのセッションに対抗してDTP方面の話をしたいです!よろしくお願いします!!
\end{quote}

\noindent と投稿して宣伝しました\cite{twitter.com:S_Shimotori_pub/status/1008257832863428609}.せっかくのiOSDCですから,iOSのコアな話をするのが王道でしょうが,そのようなネタは結局最後まで思い浮かばなかったので応募しませんでした.

こうした経緯により,皆様の応援をもって「フォントと組版の30分入門」は無事採択となりました.「iOSエンジニアに聞いて欲しいトーク(30分)」枠かつニッチ狙いですので,iOSで縦書きを試みるような実用的な話は行わないつもりです.よろしくお願いします.

\begin{flushright}
2018年7月14日 \\
しもとりしぐれ
\end{flushright}



\cleardoublepage

%------------------------------%

\pagenumbering{arabic}

\newgeometry{hmarginratio=1:1}
\part{internationalization(仮)}\label{part:internationalization}
\restoregeometry

\chapter{色々な言語(仮)}\label{chapter:internationalization/languages}
\thispagestyle{plain}

\begin{figure}[htbp]
    \begin{minipage}{0.5\hsize}
        \begin{center}
            \includegraphics[width=\linewidth]{images/ios_preferences_ltr.png}
        \end{center}
        \caption{英語}
    \end{minipage}
    \begin{minipage}{0.5\hsize}
        \begin{center}
            \includegraphics[width=\linewidth]{images/ios_preferences_rtl.png}
        \end{center}
        \caption{アラビア語}
    \end{minipage}
\end{figure}

iOS9からright-to-left言語がサポートされた.Auto Layoutではleading/trailingに対して制約をつけられるようになり,stack viewが新しく使えるようになった\cite{developer.apple.com:videos/play/wwdc2016/232/}\cite{developer.apple.com:library/archive/releasenotes/General/RN-iOSSDK-9.0/index.html}.Auto Layout,stack view,grid viewは自動的に反転される.

base internationalが有効かつAuto Layoutを使用していれば,ある程度はright-to-left言語に合った表示に自動的に切り替わる\footnote{Googleと同じ方針\cite{material.io/design/usability/bidirectionality.html}を取っているように見える.}\cite{developer.apple.com:library/archive/documentation/MacOSX/Conceptual/BPInternational/SupportingRight-To-LeftLanguages/SupportingRight-To-LeftLanguages.html}.

\begin{figure}[htbp]
    \begin{minipage}{0.5\hsize}
        \begin{center}
            \includegraphics[width=\linewidth]{images/ios_preferences_regular.png}
        \end{center}
        \caption{普通}
    \end{minipage}
    \begin{minipage}{0.5\hsize}
        \begin{center}
            \includegraphics[width=\linewidth]{images/ios_preferences_tall.png}
        \end{center}
        \caption{tall script}
    \end{minipage}
\end{figure}


\chapter{英語}\label{chapter:internationalization/languages/english}
\begin{description}
    \item[space(U+0020)]
    \item[no-break space(U+00A0)]
\end{description}

伝統的な組版では単語間のスペースに複数の文字を使用していた.デジタルフォントでは,word spaceとは単語間スペースに用いられる特定の幅の文字である\cite{docs.microsoft.com:en-us/typography/develop/character-design-standards/whitespace}.

en dash\cite{handbook_of_typography}

\begin{itemize}
    \item 括弧のようにダーシを対にして語の前後に入れ,強調する。
    \item 日時の期間,年齢の幅,ページ,値段などの範囲。
    \item 対局のもの,地点と地点,人と人など,関係や連結。
\end{itemize}

em dash\cite{handbook_of_typography}

\begin{itemize}
    \item 文と文の間,字句と字句の間に用いて,時間の経過を表す。
    \item 括弧のようにダーシを対にして文を囲み,説明や副題などを示す。
    \item 行頭に用いて,引用を表す。
    \item 語尾に用いて省略を表す。
\end{itemize}

マイナス記号は,通常キーボードで打ちやすいhyphen-minus[ハイフンマイナス]が使われているが,それとは別に減算記号としてminus sign[マイナス記号]があるので,数学の組版では注意が必要。なお,多くのプログラム等では,演算における減算記号はマイナス記号でなくハイフンマイナスが使用される\cite{handbook_of_typography}。

\chapter{アラビア語}\label{chapter:internationalization/languages/arabic}
印刷物に関して言えば,早くは10〜11世紀頃にエジブトで木版印刷が行われていた形跡があるが,イスラーム世界ではヨーロッパから活版印刷が伝わった後も,諸々の理由で近代になるまで普及しなかった.18世紀にドイツで発明されたリトグラフは,手書き写本のような字形やレイアウトを表現できるため,特にイランやパキスタンで好まれた。今日ではコンピュータによる組版が普及しているが,イスラーム世界には漢字圏同様の書道の伝統が息づいており,クルアーンやペルシア語の古典詩作品等は,美しい書体で印刷されることが多い。ユニコード標準では,アラビア文字のそれぞれについて,左右の文字と結合するかしないかといった情報を Unicode 文字データベース(UCD)内の ArabicShaping.txt ファイルによって定義している.また,ユニコードにおけるアラビア文字の扱いについて詳しくは,ユニコード標準の規格書の第 9 章を参照\cite{islamic_area_studies_resources}.

アラビア文字の印刷は、中東・イスラム地域よりもはるかに早く、ヨーロッパにおいて開始された。16世紀に入ると、キリスト教の祈祷書など、アラブ人キリスト教徒向けの出版物からアラビア文字の印刷技術は定着し、後には東洋学の発展とともに、地理書、文法書、古典文学など、さまざまな非宗教文献も刊行されるようになった。イスラム教徒が印刷技術の導入に消極的だった原因については様々な理由が挙げられている。例えば、イスラム教徒の伝統的な学問では、知識は師弟相伝の記憶によって伝えられることが重視されていたことが指摘される。18世紀に顕著となったオスマン帝国のヨーロッパ諸国に対する軍事的な弱体化を背景に、ヨーロッパ式の軍事改革と科学技術の導入が不可避となった結果として、オスマン帝国でも、印刷によって書物を大量生産し、知識を普及する必要性がようやく認識されたことになる。文字の美観の問題も見逃すことはできない。19世紀頃までヨーロッパで使用されていた活字はのっぺりとして不格好であり、美しい筆跡の写本を愛好するイスラム教徒の審美眼に適うものではなかった7。イスラム社会の伝統的な学問分野では、100年前のミュテフェッリカの時代には印刷が禁止されていた宗教関係が出版の対象に含まれることが目を引く。ブーラーク以後、19世紀にはイスラム教徒が経営し、アラビア文字で印刷を行う出版事業が中東全域に広がった\cite{rnavi.ndl.go.jp:asia/entry/bulletin7-3-2.php}。

\chapter{タイ語}\label{chapter:internationalization/languages/thai}
タイ語はタイ王国の公用語.タイ全土で話されている.使用人口はタイ国内に約6200万人\cite{www.coelang.tufs.ac.jp:mt/th/}.

ひとつの音節が,子音文字,母音記号,声調記号などの記号の組み合わせで表示される.子音字を中心に母音記号を上下左右に配置する音節文字である.子音字の文字幅は文字によって大きく異なる(=プロポーショナル).発音順とグリフの並び順は一致せず,音素と文字のグリフが1対1対応でないことがある\cite{www.aa.tufs.ac.jp:mmine/lecture/lec03/TLKChar/lecTLK03.htm}.

通常,タイ文字組版の文章内ではスペースを空けずに組むが,アラビア数字や英数字が文中に入るときは前後にスペースを入れる.組み方向は横組み,行揃えは均等配置(最終行左揃え)が基本だが左揃えやセンター揃えが使われることもある.単語の間では改行を行わない(ハイフネーションに相当するものがない)ので改行自体が少ない方が好まれる.文字の上下に記号がつくので十分に行間をとる必要がある.公文書ではインデントを行って組むこととされる.句読点のような文字はない\cite{www.morisawa.co.jp:fonts/multilingual/typesetting/pdf/all_multilingualtypesetting_1802.pdf}.


\newgeometry{hmarginratio=1:1}
\part{iOS}\label{part:ios}
\restoregeometry

\chapter{iOSにおけるRTLレイアウト}\label{chapter:ios/rtl}
\thispagestyle{plain}

\begin{figure}[htbp]
    \begin{minipage}{0.5\hsize}
        \begin{center}
            \includegraphics[width=\linewidth]{images/ios_preferences_ltr.png}
        \end{center}
        \caption{英語}
    \end{minipage}
    \begin{minipage}{0.5\hsize}
        \begin{center}
            \includegraphics[width=\linewidth]{images/ios_preferences_rtl.png}
        \end{center}
        \caption{アラビア語}
    \end{minipage}
\end{figure}

iOS9からright-to-left言語がサポートされた.Auto Layoutではleading/trailingに対して制約をつけられるようになり,stack viewが新しく使えるようになった\cite{developer.apple.com:videos/play/wwdc2016/232/}\cite{developer.apple.com:library/archive/releasenotes/General/RN-iOSSDK-9.0/index.html}.Auto Layout,stack view,grid viewは自動的に反転される.

base internationalが有効かつAuto Layoutを使用していれば,ある程度はright-to-left言語に合った表示に自動的に切り替わる\footnote{Googleと同じ方針\cite{material.io/design/usability/bidirectionality.html}を取っているように見える.}\cite{developer.apple.com:library/archive/documentation/MacOSX/Conceptual/BPInternational/SupportingRight-To-LeftLanguages/SupportingRight-To-LeftLanguages.html}.

\begin{figure}[htbp]
    \begin{minipage}{0.5\hsize}
        \begin{center}
            \includegraphics[width=\linewidth]{images/ios_preferences_regular.png}
        \end{center}
        \caption{普通}
    \end{minipage}
    \begin{minipage}{0.5\hsize}
        \begin{center}
            \includegraphics[width=\linewidth]{images/ios_preferences_tall.png}
        \end{center}
        \caption{tall script}
    \end{minipage}
\end{figure}



\chapter{TextKit解説(仮)}\label{chapter:ios/textKit}
\thispagestyle{plain}

\begin{figure}[htbp]
    \begin{minipage}{0.5\hsize}
        \begin{center}
            \includegraphics[width=\linewidth]{images/ios_preferences_ltr.png}
        \end{center}
        \caption{英語}
    \end{minipage}
    \begin{minipage}{0.5\hsize}
        \begin{center}
            \includegraphics[width=\linewidth]{images/ios_preferences_rtl.png}
        \end{center}
        \caption{アラビア語}
    \end{minipage}
\end{figure}

iOS9からright-to-left言語がサポートされた.Auto Layoutではleading/trailingに対して制約をつけられるようになり,stack viewが新しく使えるようになった\cite{developer.apple.com:videos/play/wwdc2016/232/}\cite{developer.apple.com:library/archive/releasenotes/General/RN-iOSSDK-9.0/index.html}.Auto Layout,stack view,grid viewは自動的に反転される.

base internationalが有効かつAuto Layoutを使用していれば,ある程度はright-to-left言語に合った表示に自動的に切り替わる\footnote{Googleと同じ方針\cite{material.io/design/usability/bidirectionality.html}を取っているように見える.}\cite{developer.apple.com:library/archive/documentation/MacOSX/Conceptual/BPInternational/SupportingRight-To-LeftLanguages/SupportingRight-To-LeftLanguages.html}.

\begin{figure}[htbp]
    \begin{minipage}{0.5\hsize}
        \begin{center}
            \includegraphics[width=\linewidth]{images/ios_preferences_regular.png}
        \end{center}
        \caption{普通}
    \end{minipage}
    \begin{minipage}{0.5\hsize}
        \begin{center}
            \includegraphics[width=\linewidth]{images/ios_preferences_tall.png}
        \end{center}
        \caption{tall script}
    \end{minipage}
\end{figure}



\chapter{iOSで縦書きを試みる(仮)}\label{chapter:ios/vertical}
\thispagestyle{plain}

\begin{figure}[htbp]
    \begin{minipage}{0.5\hsize}
        \begin{center}
            \includegraphics[width=\linewidth]{images/ios_preferences_ltr.png}
        \end{center}
        \caption{英語}
    \end{minipage}
    \begin{minipage}{0.5\hsize}
        \begin{center}
            \includegraphics[width=\linewidth]{images/ios_preferences_rtl.png}
        \end{center}
        \caption{アラビア語}
    \end{minipage}
\end{figure}

iOS9からright-to-left言語がサポートされた.Auto Layoutではleading/trailingに対して制約をつけられるようになり,stack viewが新しく使えるようになった\cite{developer.apple.com:videos/play/wwdc2016/232/}\cite{developer.apple.com:library/archive/releasenotes/General/RN-iOSSDK-9.0/index.html}.Auto Layout,stack view,grid viewは自動的に反転される.

base internationalが有効かつAuto Layoutを使用していれば,ある程度はright-to-left言語に合った表示に自動的に切り替わる\footnote{Googleと同じ方針\cite{material.io/design/usability/bidirectionality.html}を取っているように見える.}\cite{developer.apple.com:library/archive/documentation/MacOSX/Conceptual/BPInternational/SupportingRight-To-LeftLanguages/SupportingRight-To-LeftLanguages.html}.

\begin{figure}[htbp]
    \begin{minipage}{0.5\hsize}
        \begin{center}
            \includegraphics[width=\linewidth]{images/ios_preferences_regular.png}
        \end{center}
        \caption{普通}
    \end{minipage}
    \begin{minipage}{0.5\hsize}
        \begin{center}
            \includegraphics[width=\linewidth]{images/ios_preferences_tall.png}
        \end{center}
        \caption{tall script}
    \end{minipage}
\end{figure}


\section{{\sf UILabel}幅の縮小による縦書き}\label{chapter:ios/vertical/numberOfLines}
最も簡単に縦書きにする方法は,{\sf UILabel}の幅をなるべく小さくすることで1文字ずつ改行させる方法である(プログラム\ref{lst:ios/vertical/uiLabelWidth/example}).

\begin{lstlisting}[language=swift,caption={\sf UILabel}の幅を縮小することによる縦書きの実現,label=lst:ios/vertical/uiLabelWidth/example]
let label = UILabel()
label.text = "30 minutes Introduction to Font and Typesetting"
label.font = UIFont(name: "Courier", size: 17)
label.numberOfLines = 0
label.frame = CGRect(x: 0, y: 0, width: 17, height: 500)
\end{lstlisting}

システムフォントのSan Francisco\footnote{正確にはSan Francisco Pro DisplayもしくはSan Francisco Pro Text.20pt以上でDisplayに,20pt未満でTextになる\cite{developer.apple.com:design/human-interface-guidelines/ios/visual-design/typography/}.}は等幅ではないため,Courierなどの等幅フォントにすることである程度綺麗に見せることができる.いずれにせよ,全角文字と半角文字が混ざった文字列や拗音は正しい位置に表示されない.



\newgeometry{hmarginratio=1:1}
\part{パート}\label{part:draft}
\restoregeometry

\chapter{チャプター}\label{chapter:draft}
\thispagestyle{plain}
チャプター

%------------------------------%

\newgeometry{hmarginratio=1:1}
\begin{appendices}
\restoregeometry
\chapter{用語集}
% TODO: fix
\thispagestyle{fancy}
用語(あとで消す)

モリサワとW3C\cite{www.w3.org:TR/2011/WD-jlreq-20111129/ja/}あたりから引用中.

\begin{description}
    \item[Advancement]
    \item[Alignment] alignmentとdirectionalityは異なるものを指すので注意\cite{developer.apple.com:videos/play/wwdc2016/232/}.
    \item[Ascent]
    \item[Baseline(並び線)]
    \item[base writing direction] 最初に登場したstrong characterで決定される\cite{developer.apple.com:videos/play/wwdc2016/232/}\cite{unicode.org:reports/tr9/}.
    \item[bidirectional algorithm(双方向アルゴリズム)] bidirectional textを扱うためのアルゴリズム\cite{www.w3.org:International/articles/inline-bidi-markup/uba-basics}.bidi algorithmとも.このアルゴリズムによる脆弱性が見つかったことがある\cite{jvndb.jvn.jp:ja/contents/2010/JVNDB-2010-002420.html}.
    \item[bidirectional text(双方向テキスト)] 左横書き(LTR)と右横書き(RTL)が混在するテキストのこと\cite{www.w3.org:International/articles/inline-bidi-markup/uba-basics}.bidiとも.
    \item[block direction(行送り方向)]
    \item[bold(太字)]
    \item[Boldface]
    \item[Bounding rectangle]
    \item[break (a line)(分割)]
    \item[Cap height]
    \item[centering(中央そろえ)]
    \item[character advance(字幅)]
    \item[character frame(外枠)]
    \item[characters not starting line(行頭禁則文字)]
    \item[CJK] 中国日本韓国,特に中国語・日本語・韓国語のこと.UnicodeのブロックにもCJKが散見される\cite{unicode.org:Public/UNIDATA/Blocks.txt}.
    \item[composition(組版)]
    \item[descender line(ディセンダライン)]
    \item[Descent]
    \item[DTP(desktop publishing)]1984年1月、Apple社から初代Macintoshが発売される。プラットフォームとして様々な周辺機器やソフトウェアが生み出された。ただし初期のMacは本格的なDTPを行うにはスペックが厳しく、DTP業界が急拡大するのは1987年発売のMacintosh II頃からである。1985年5月、Apple純正のレーザープリンターであるApple LaserWriterが発売される。LaserWriterプリンターは、アドビ社の開発したページ記述言語・PostScript技術を用いた「Adobe PostScriptフォント」がROMメモリに組み込まれており、これによって画面に表示されているものをそのままに印刷することが可能となる「WYSIWYG」を実現したほか、プリンターにPostScriptフォントを搭載している限りはコンピュータとプリンターの組み合わせが変わっても出力結果を維持するという「デバイスインディペンデント」(使用機器に依存しない)な性質を実現していた。1985年7月、Macintoshプラットフォームにおける最初の実用的なDTPアプリケーションとなるアルダス社のPagemakerが発売される。これによってDTP環境が実現された。(ここまでウィキペディア.要出典.新詳説DTP基礎)
    \item[Emoji(絵文字)]
    \item[even inter-character spacing(均等割り)]
    \item[even tsumegumi(均等詰め)]
    \item[face tsumegumi(字面詰め)]
    \item[fixed inter-character spacing(アキ組)]
    \item[fixed-width(モノスペース)]
    \item[Font(フォント)]
    \item[Font fallback]
    \item[Font Family]
    \item[full-width(全角)]
    \item[Glyph(グリフ)]字体とほぼ同義語ですが、記述記号やスペースなども含めたものを指します。慣用的にはデータとしての字体を指す場合に使われることもあります。これらの文字と記号類を集めたものがグリフセットと呼ばれるもので、これは文字セットや文字コレクションとほぼ同義と考えてよいでしょう。 A glyph is the smallest displayable unit in a font. A glyph may represent one character, more than one character, or part of a character. The mapping of characters to glyphs is not simple—it can be many-to-many. In addition, the order and position of glyphs in a line is complex\cite{developer.apple.com:library/archive/documentation/MacOSX/Conceptual/BPInternational/InternationalizingYourCode/InternationalizingYourCode.html}.
    \item[half em(二分)]
    \item[half em space(二分アキ)]
    \item[half-width]
    \item[horizontal writing mode(横組)]
    \item[Hyphenation(ハイフネーション)]
    \item[isolates support]\cite{developer.apple.com:videos/play/wwdc2016/232/}
    \item[Italic angle]
    \item[inline direction(字詰め)]
    \item[inseparable characters rule(分離禁止)]
    \item[JIS2004]
    \item[JIS X 0213:2004]
    \item[JIS文字セット]
    \item[Kerning]
    \item[Leading]
    \item[letter face(字面)]
    \item[letterpress printing(活字組版)]
    \item[Ligature]
    \item[line adjustment(行の調整処理)]
    \item[line adjustment by hanging punctuation(ぶら下げ組)]
    \item[line adjustment by inter-character space expansion(追出し処理)]
    \item[line adjustment by inter-character space reduction(追込み処理)]
    \item[Line breaking]
    \item[line breaking rules(禁則処理)]
    \item[line end(行末)]
    \item[line end alignment(行末そろえ)]
    \item[line head indent(字下げ)]
    \item[line-end prohibition rule(行末禁則文字)]
    \item[line feed(行送り)]
    \item[Line gap(行間)]
    \item[line head(行頭)]
    \item[line head alignment(行頭そろえ)]
    \item[Line height]
    \item[line length(行長)]
    \item[line-start prohibition rule(行頭禁則)]
    \item[matrix(母型)]
    \item[mixed text composition(混植)]
    \item[Monospace(等幅)]
    \item[new column(改段)]
    \item[new recto(改丁)]
    \item[number of characters per line(字詰め)]
    \item[one em space(全角アキ)]
    \item[one third em(三分)]
    \item[one third em space(三分アキ)]
    \item[Open Type Font]
    \item[Orphan avoidance]
    \item[page break(改ページ)]
    \item[paragraph(段落)]
    \item[paragraph break(改行)]
    \item[paragraph format(段落整形)]
    \item[PDF]
    \item[Post Script]
    \item[printing types(活字)]
    \item[proportional(プロポーショナル)]
    \item[quarter em(四分)]
    \item[quarter em space(四分アキ)]
    \item[quarter em width(四分角)]
    \item[Serif]
    \item[single line alignment method(そろえ)]
    \item[solid setting(ベタ組)]
    \item[space(アキ)]
    \item[stem]
    \item[tab setting(タブ処理)]
    \item[tall script]アクセント記号や声調記号の関係でline heightが他の言語の文字より高くなっているもの.文字全体が表示されるように余裕を持っておく必要がある\cite{developer.apple.com:videos/play/wwdc2016/201/}.この手の文字を太字にすると重くなってしまうので太字にはしないほうがいいらしい\cite{material.io:design/typography/language-support.html}.
    \item[text direction(組方向)]
    \item[Tightening]
    \item[Tracking]
    \item[True Type Font]
    \item[Truncation]
    \item[Typeface(書体)]
    \item[type-picking(文選)]
    \item[typesetting(組版)] 原稿に基づいて活字を組み,最終的には版を作るまでの作業および完成した版.手動,機械さらにはコンピュータであっても,組版という表現は使われている.手動,機械さらにはコンピュータであっても,組版という表現は使われている\cite{lis_dictionary}.1970年代からはコンピュータによる組版が普及した\cite{www.printing-museum.org:communication/column/pdf/column_6.pdf}.
    \item[Typography]
    \item[unbreakable characters rule(分割禁止)]
    \item[vertical writing(縦書き)] 日本語ではおなじみ縦書き.アジア圏でよく見られるっぽい.top to bottomとも\cite{eikaiwa.dmm.com:uknow/questions/29852/}\cite{www.w3.org:International/questions/qa-scripts}.CSSの{\sf writing-mode}プロパティでは,CJK用に{\sf vertical-rl}が,モンゴル語用に{\sf vertical-lr}が用意されている\cite{www.w3.org:International/articles/vertical-text/}.
    \item[vertical writing mode(縦組)]
    \item[Weight(ウェイト)]
    \item[widow(ウィドウ)]
    \item[widow adjustment(段落末尾処理)]
    \item[WYSIWYG]
    \item[X-height]
    \item[かべ]
    \item[基本版面]
    \item[ゴシック体]
    \item[字形]
    \item[字体]
    \item[字取り]
    \item[字面]
    \item[字面枠]
    \item[書体]
    \item[縦中横]
    \item[詰め組]
    \item[天付き]
    \item[同行見出し]
    \item[フォント]
    \item[プロポーショナルメトリクス]
    \item[ペアカーニング情報]
    \item[ベタ組み]
    \item[丸ゴシック体]
    \item[明朝体]
\end{description}


\end{appendices}

\printbibheading[heading=bibintoc,title={参考文献}]
\thispagestyle{plain}
\printbibliography[heading=subbibintoc,type=online,title={Online}]
\printbibliography[heading=subbibintoc,type=book,title={Book}]

\printindex

\end{document}

